% Ubah judul dan label berikut sesuai dengan yang diinginkan.
\section{Introduction}
\label{sec:pendahuluan}

% Ubah paragraf-paragraf pada bagian ini sesuai dengan yang diinginkan.

Face detection algorithm with high accuracy and fast performance is needed
at this time. This is inseparable from the need for face detection algorithms to
implemented as an identity authentication tool and a biometric-based security system.
Biometrics is an automatic identification system of individuals based on physical and/or physical characteristics
behavioral characteristics in individuals such as fingerprints, faces, irises, or voice \citep{doi:10.1080/09505431.2018.1519534}.
Biometrics were chosen because they have a unique feature, namely they cannot be transferred (non-transferable characteristic) because biometrics
recording physical conditions and/or behavioral conditions in individuals, it is more
reliable compared to legacy authentication systems such as passwords or PINs that only record numbers
or certain letters \citep{1597098}. Therefore, at this time many biometric systems have been developed
on the security system to switch from authentication systems such as passwords and PINs. Compared to the type of biometric that does the recording
Based on physical characteristics, Biometrics using the face is currently still an area
research since 1960 with a wide scope for continuous improvement \citep{zufar}.

More and more methods are being created, making face detection systems more sophisticated today.
There are several methods that can be used according to the required conditions. But basically
real-time conditions with fast computing processes are the main goals of developments
the method. The Deep Learning method that is widely used and developed in face detection algorithms is
CNN based method. CNN is a method inspired by Neural Network (NN) which can solve
a problem with passing an input X into the Convolutional filter circuit and other nonlinear computations \citep{lecun1989handwritten}.
The development of the most recent CNN-based face detection method is one of the facial detection methods based on the feature pyramid. Fundamental difference
of the feature pyramid based face detection method when compared to other face detection methods is the feature pyramid-based face detection method
perform the process of obtaining a feature map with a high semantic level using the feature pyramid network architecture \citep{lin2017feature}. 
The advantage of the feature pyramid-based face detection method is that it can accurately detect the face area of each individual 
in the condition of many faces in one image or frame.

With many methods that have been developed, the face detection algorithm still has challenges from aspects of the detection process
These aspects are the condition of facial variations that vary between individuals, and the size of the face that depends on the distance and search space \citep{Li_2015_CVPR}.
Based on this, in this final project, an evaluation of the performance of the face detection system using the feature pyramid-based face detection method will be carried out. System
face detection based on feature pyramid will be tested using several scenarios of facial variations in different individuals and environments,
So that the output of this final project obtained the best conditions for detecting.
