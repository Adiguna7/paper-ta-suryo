% % Ubah judul dan label berikut sesuai dengan yang diinginkan.
\section{Conclusions and Recommendations}
\label{sec:kesimpulan}

\subsection{Conclusions}
Based on the results of the tests that have been carried out, several conclusions can be drawn as follows:

\begin{enumerate}

  \item A face detection system based on feature pyramid with backbone mobilenet0.25 is able to provide
  up to seven times faster processing time and higher FPS value
  up to two times better than face detection systems based on 
  feature pyramid with \emph{backbone resnet50}.
  Meanwhile, a system based on \emph{backbone resnet50} is able to provide a higher mAP value of up to $20\%$
  rather than a backbone mobilenet0.25 based system.

  \item The difference in the height of the camera placement also affects the value of
  mAP obtained, where when using a camera that is placed parallel to the object's face,
  mAP value obtained is higher.

  \item The angle of the camera greatly affects the performance of backbone mobilenet0.25, Backbone mobilenet0.25
  can only detect well in the corners of $50^\circ$ and $60^\circ$. Whereas backbone resnet50 can detect
  performs well in almost all conditions with longer processing times and detection FPS.

  \item When compared to YOLO and SSD, the face detection system based on feature pyramid is better in almost all aspects
  one of them if YOLO and SSD can't detect more than three obstacles \citep{nugrohoevalution2016}. Face detection system
  based on feature pyramid can detect it with good accuracy.

  \item The time of taking video data when using a face detection system based on feature pyramid has no effect on the mAP value.
  However, if given additional artificial light and the camera is placed parallel, then the mAP value of the two backbone becomes better.

  \item In a test based on the number of objects, a face detection system based on feature pyramid can detect all faces without any errors.
  better when compared to YOLO and SSD from previous studies which got an error of $3\%$ \citep{nugrohoevalution2016}.


\end{enumerate}


\subsection{Recomendations}

For further development of this final project, there are several suggestions that can be made, including:

\begin{enumerate}[nolistsep]

  \item Try another backbone that fits the desired condition

  \item Try compare CNN and DCN on context-module face detection system based on feature pyramid to see if when
  the system will work faster.

  \item Implementing a facial detection system based on feature pyramid into an implementation of a facial recognition system, such as surveillance of theft or so on.
  
  \item Implementing a face detection system based on feature pyramid into an application implementation, which can be used using the user interface.

\end{enumerate}
% % Ubah paragraf-paragraf pada bagian ini sesuai dengan yang diinginkan.

% \lipsum[21-23]
